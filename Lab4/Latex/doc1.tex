\documentclass[a4paper, 12pt]{article}
\usepackage{hyperref}
\usepackage{graphicx}

\begin{document}

\begin{titlepage}

	\begin{center}
	\textsc{\large Facultatea Calculatoare, Informatica si Microelectronica}\\[0.5cm]
	\textsc{\large Universitatea Tehnica a Moldovei}\\[1.2cm]
	\vspace{25mm}

	\textsc{\Large Medii Interactive de Dezvoltare a Produselor Soft}\\[0.5cm]
  	\textsc{\large Lucrarea de laborator\#3}\\[0.5cm] 

	\newcommand{\HRule}{\rule{\linewidth}{0.5mm}} 
	\vspace{10 mm}

  	\HRule \\[0.4cm]

 	 { \LARGE \bfseries Web development  }\\[0.4cm] 

  	\HRule \\[1.5cm]

	\vspace{30mm}

	\begin{minipage}{0.4\textwidth}
	\begin{flushleft} \large
	\emph{Autor:} \\
	Crivenco \textsc{Vladislav}\\
	\end{flushleft}
	\end{minipage}
      	\begin{minipage}{0.4\textwidth}

      	\begin{flushright} \large

      	\emph{lector asistent:} \\

      	Irina \textsc{Cojanu} \\ % Supervisor's Name 

     	 \end{flushright}

      	\end{minipage}\\[4cm]



     	\vspace{5 mm}

	\vfill

	\end{center}

\end{titlepage}

\section{Obiectivele laboratorului}
\subsection{Realizarea unui simplu Web Site personal}
\subsection{Familiarizarea cu HTML si CSS}
\subsection{Interactiuni Javascript}

\section{Lista de tascuri implementate}
	\begin{enumerate}
	\item Realizeaza un mini site cu 3 pagini statice
	\item Pastrarea  informatiei intr-o baza de date
	\item Informatia este incarcata dinamic pe pagina
	\end{enumerate}

\section{Analiza lucrarii de laborator}
\href{https://github.com/VladislavCrivenco/MIDPS}{Repozitoriul pe Github}

\subsection{Realizeaza un mini site cu 3 pagini statice}

In acest laborator a fost realizat un site in care useru isi poate adauga albume si muzica iubita, inclusiv fisiere audio pentru a le putea asculta online.
Deci prima pagina este cea de inregistrare
\linebreak[4]
\includegraphics*[width=10cm, height=15cm]{reg}

Si respectiv pagina de logare

\includegraphics*[width=10cm, height=15cm]{log}

O pagina pentru adaugarea unui album nou

\includegraphics*[width=10cm, height=10cm]{album}

Si o pagina de adaugare a unei piese muzicale in album

\includegraphics*[width=10cm, height=13cm]{song}

Deasemnea situl are un meniu de navigare 

\includegraphics*[width=15cm, height=2cm]{menu}

\subsection{Pastrarea informatiei intr-o baza de date}
Framework-ul automat leaga proiectul cu o baza de date MySql,
in care creeaza tabele pentru fiecare din modele(Album si Song)

\includegraphics*[width=15cm, height=16cm]{model}

Daca ne logam ca admin ,putem sa modificam tot ce se contine in baza de date 

\includegraphics*[width=15cm, height=16cm]{admin}

\newpage
\section{Concluzie}
In urma acestui laborator am facut cunostinta cu procesul de dezvoltare a unui site utilizind un frame-work.Avind 0 cunostinte in acest domeniu, acum inteleg structura unui site, in special partea de back-end.Stiu cum sa leg serverul cu o baza de date si sa fac manipulari la nivel esential cu ea.
Deasemenea am incercat Bootstrap care mia usurat mult lucrul la crearea meniului si partea de dezign.
\clearpage

\end{document}