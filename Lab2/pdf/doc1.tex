\documentclass[a4paper, 12pt]{article}
\usepackage{hyperref}
\usepackage{graphicx}

\begin{document}

\begin{titlepage}

	\begin{center}
	\textsc{\large Facultatea Calculatoare, Informatica si Microelectronica}\\[0.5cm]
	\textsc{\large Universitatea Tehnica a Moldovei}\\[1.2cm]
	\vspace{25mm}

	\textsc{\Large Medii Interactive de Dezvoltare a Produselor Soft}\\[0.5cm]
  	\textsc{\large Lucrarea de laborator\#2}\\[0.5cm] 

	\newcommand{\HRule}{\rule{\linewidth}{0.5mm}} 
	\vspace{10 mm}

  	\HRule \\[0.4cm]

 	 { \LARGE \bfseries GUI Developement  }\\[0.4cm] 

  	\HRule \\[1.5cm]

	\vspace{30mm}

	\begin{minipage}{0.4\textwidth}
	\begin{flushleft} \large
	\emph{Autor:} \\
	Crivenco \textsc{Vladislav}\\
	\end{flushleft}
	\end{minipage}
      	\begin{minipage}{0.4\textwidth}

      	\begin{flushright} \large

      	\emph{lector asistent:} \\

      	Irina \textsc{Cojanu} \\ % Supervisor's Name 

     	 \end{flushright}

      	\end{minipage}\\[4cm]



     	\vspace{5 mm}

	\vfill

	\end{center}

\end{titlepage}

\section{Obiectivele laboratorului}
Realizeaza un simplu GUI Calculator si divizare proiectului in doua module - Interfata grafica(Modul GUI) si Modulul de baza(Core Module).

\section{Lista de tascuri implementate}
	\begin{enumerate}
	\item Realizeaza un simplu GUI calculator care suporta urmatoare functii: +, -, /, *, putere, radical, InversareSemn(+/-), operatii cu numere zecimale.
	\item Divizare proiectului in doua module - Interfata grafica(Modul GUI) si Modulul de baza(Core Module).
	\end{enumerate}

\section{Analiza lucrarii de laborator}
\href{https://github.com/VladislavCrivenco/MIDPS}{Repozitoriul pe Github}

\subsection{Realizeaza un simplu GUI calculator}

Pentru a dezvolta un GUI calculator am folosit limbajul C\# si Visual Studio 2012 ca IDE ,deoarece este foarte simplu de aranjat partea grafica a calculatorului (adaugarea butonurilor, TextBoxului) si respectiv modificarea setarilor fiecarui element grafic.
\newpage
\includegraphics*[width=15cm, height=16cm]{GUI}

Visual Studio automat genereaza listening la butoane prin crearea unui corp de functie. Noua ne ramine doar sa scriem logica programului in aceasta functie.

\includegraphics*{listen}

\subsection{Divizare proiectului in doua module }
Aici tot lucrul practic a fost realizat de insasi IDE, Visual Studio automat a creat mai multe file-uri.

File Designer.cs contine declararea tuturor elemente de GUI
si respectiv proprietatile lor (dimensiunea, locatia, stilul).

\includegraphics*[width=15cm, height=16cm]{gui_gen}

File Form1.cs include logica programului , calcularea rezultatului si managmentul erorilor.

\includegraphics*[width=15cm, height=16cm]{logic_gen}

\newpage
\section{Concluzie}
In urma acestui laborator am facut cunostinta cu Window Forms si cu elemente de baza din C\#.Am insusit cum pot adauga elemente grafice cu usurinta in aplicatia mea, si am inteles cum pot modifica majoritatea din proprietatile fiecarui element.
Utilizarea unui limbaj nou nu a creat dificultati din cauza banalitatii functionalului aplicatiei ,si datorita faptului ca acest limbaj se aseamana tare cu toate limbajele din familia C.
\clearpage

\end{document}